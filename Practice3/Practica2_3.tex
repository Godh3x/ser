\documentclass[11pt]{article}
\usepackage[utf8]{inputenc}

%%% PAGE DIMENSIONS
\usepackage{geometry}
\geometry{a4paper}

\usepackage{graphicx}

%%% PACKAGES
\usepackage{booktabs}
\usepackage{paralist}
\usepackage{verbatim}
\usepackage{subfig}
\usepackage{chngcntr}
\usepackage{tikz}
\usepackage[colorlinks = true,
            linkcolor = black,
            urlcolor  = blue,
            citecolor = blue,
            anchorcolor = blue]{hyperref}
\usepackage[spanish]{cleveref}
\usepackage{placeins}
\usepackage{float}

%%% HEADERS & FOOTERS
\usepackage{fancyhdr}
\pagestyle{fancy}
\renewcommand{\headrulewidth}{0pt}
\lhead{}\chead{}\rhead{}
\lfoot{}\cfoot{\thepage}\rfoot{}

%%% SECTION TITLE APPEARANCE
\usepackage{sectsty}
\allsectionsfont{\sffamily\mdseries\upshape}

%%% ToC (table of contents) APPEARANCE
\usepackage[nottoc,notlof,notlot]{tocbibind} % Put the bibliography in the ToC
\usepackage[titles,subfigure]{tocloft} % Alter the style of the Table of Contents
\renewcommand{\cftsecfont}{\rmfamily\mdseries\upshape}
\renewcommand{\cftsecpagefont}{\rmfamily\mdseries\upshape} % No bold!


\graphicspath{ {images/} }

\counterwithin*{figure}{section}
\counterwithin*{figure}{subsection}
\counterwithin*{figure}{subsubsection}

\counterwithin*{table}{section}
\counterwithin*{table}{subsection}
\counterwithin*{table}{subsubsection}

\addtolength{\cftfignumwidth}{2em}

\renewcommand{\thefigure}{
  \ifnum\value{subsection}=0
    \thesection.\arabic{figure}
  \else
    \ifnum\value{subsubsection}=0
      \thesubsection.\arabic{figure}
    \else
      \thesubsubsection.\arabic{figure}
    \fi
  \fi
}

\renewcommand{\thetable}{
  \ifnum\value{subsection}=0
    \thesection.\arabic{table}
  \else
    \ifnum\value{subsubsection}=0
      \thesubsection.\arabic{table}
    \else
      \thesubsubsection.\arabic{table}
    \fi
  \fi
}

%%% END Article customizations

%%% The "real" document content comes below...

\title{\Large Seguridad en Redes\\Practica 2.3}
\author{David Antuña Rodríguez\\Javier Carrión García}
\date{}

\begin{document}
  \raggedright

  \maketitle
  \newpage

  \section{OpenSSL}
    \subsection{Creación de una CA}
      \par
      Si intentamos utilizar otros directorios dará error a no ser que se modifique el fichero de
      configuración, concretamente los datos de la sección [CA\_default].

      \par
      La clave privada se ha almacenado en \textit{demoCA/private/cakey.pem} y el certificado en
      \textit{demoCA/cacert.pem}. Como hemos empleado la opción -new la clave generada es de tipo
      RSA y su tamaño viene especificado en el fichero de configuracion, valor de default\_bits
      en la sección [req], por defecto son 2048 bits.

      \par
      \textbf{Comandos}\\
      mkdir demoCA\\
      mkdir demoCA/newcerts\\
      mkdir demoCA/private\\
      touch demoCA/index.txt\\
      echo 01 $>$ demoCA/serial\\
      echo 01 $>$ demoCA/crlnumber\\
      openssl req -x509 -new -days 3650 -keyout demoCA/private/cakey.pem -out demoCA/cacert.pem

    \subsection{Creación de solicitudes de firma de certificado}
      \par
      El algoritmo por defecto se encuentra en la variable default\_md del fichero de configuración,
      en nuestro caso hemos empleado la opción -new que crea una clave rsa nueva y la ha almacenado
      en userkey1.pem y userkey2.pem, también podriamos haber utilizado una que tuvieramos
      previamente con la opción -key.

      \par
      \textbf{Comandos}\\
      openssl req -new -keyout userkey1.pem -out usercsr1.pem \hspace{10mm}CN: usuario1\\
      openssl req -in usercsr1.pem -noout -text $>$  usercsr1.txt\\
      openssl req -verify -in usercsr1.pem\\
      openssl req -new -keyout userkey2.pem -out usercsr2.pem \hspace{10mm}CN: usuario2\\
      openssl req -in usercsr2.pem -noout -text $>$  usercsr2.txt\\
      openssl req -verify -in usercsr2.pem

    \subsection{Creación y verificación de certificados}
      \par
      \textbf{Comandos}\\
      openssl ca -in usercsr1.pem -out usercert1.pem\\
      openssl verify -CAfile demoCA/cacert.pem usercert1.pem\\
      openssl ca -in usercsr2.pem -out usercert2.pem\\
      openssl verify -CAfile demoCA/cacert.pem usercert2.pem

    \subsection{Consulta y manipulación de certificados}
      \par
      \textbf{Comandos}\\
      openssl x509 -in usercert1.pem -noout -text $>$ usercert1.txt\\
      openssl x509 -in usercert1.pem -noout -pubkey $>$ usercert1pubkey.txt\\
      openssl x509 -in usercert2.pem -noout -text $>$ usercert2.txt\\
      openssl x509 -in usercert2.pem -noout -pubkey $>$ usercert2pubkey.txt
      openssl x509 -in demoCA/cacert.pem -noout -text $>$ cacert.txt\\
      openssl x509 -in demoCA/cacert.pem -noout -pubkey $>$ cacertpubkey.txt\\
      PEM\\
      openssl x509 -in usercert1.pem -out usercert1.der -outform DER\\
      openssl x509 -in usercert1.der -inform DER -out usercert1.pem\\
      openssl pkcs12 -export -in usercert1.pem -inkey userkey1.pem -out usercert1.p12

    \subsection{Revocación de certificados}
      \par
      \textbf{Comandos}\\
      openssl ca -revoke usercert2.pem\\
      openssl ca -gencrl -out crl.pem\\
      openssl crl -in crl.pem -noout -text $>$ crl.txt\\
      openssl crl -CAfile demoCA /cacert.pem -in crl.pem\\
      openssl verify -crl\_check -CAfile demoCA/cacert.pem -CRLfile crl.pem usercert2.pem

  \section{GnuPG}
    \subsection{Firma de claves (Web of trust)}
      \par
      C será valida porque aunque desconocemos la validez de C si conocemos la de B, y nos
      fiamos de ella, y la clave C ha sido firmada por B.
      \par
      \textbf{Comandos}\\
      gpg2 $--$gen-key \hspace{10mm} nombre: claveA, psw: seguridad\\
      gpg2 $--$import pubkeysSER.gpg\\
      gpg2 $--$sign-key $--$local-user idA idB\\
      gpg2 $--$sign-key $--$local-user idB idC\\
      gpg2 $--$edit-key idB trust quit \hspace{10mm} Nivel 4\\
      gpg2 $--$edit-key idC trust quit \hspace{10mm} Nivel 1\\
      gpg2 $--$check-trustdb $>$ check\_trustdb.txt\\
      gpg2 $--$list-options show-uid-validity --list-keys $>$ validity.txt\\

      \bigskip
      \par
      En el segundo caso C no será válida porque B es dudosa y para validar una clave con firmas
      dudosas son necesarias al menos 3, también debe cumplir que el camino que lleva hasta nuestra
      firma es menor de 5, como hemos firmado B con la clave A esto se cumple pero estas condiciones
      no son excluyentes.
      \par
      \textbf{Comandos}\\
      gpg2 $--$edit-key idB trust quit \hspace{10mm} Nivel 3\\
      gpg2 $--$check-trustdb $>$ check\_trustdb2.txt\\
      gpg2 $--$list-options show-uid-validity --list-keys $>$ validity2.txt\\

      \bigskip
      \par
      Ahora C vuelve a ser válida porque como hemos explicado antes ha sido firmada por 3 claves
      dudosas (B, D y E) y ademas el camino a nuestra clave es menor de 5.
      \par
      \textbf{Comandos}\\
      gpg2 $--$sign-key $--$local-user idA idD\\
      gpg2 $--$sign-key $--$local-user idA idE\\
      gpg2 $--$edit-key idD trust quit \hspace{10mm} Nivel 3\\
      gpg2 $--$edit-key idE trust quit \hspace{10mm} Nivel 3\\
      gpg2 $--$sign-key $--$local-user idD idC\\
      gpg2 $--$sign-key $--$local-user idE idC\\
      gpg2 $--$check-trustdb $>$ check\_trustdb3.txt\\
      gpg2 $--$list-options show-uid-validity --list-keys $>$ validity3.txt\\
    \end{document}
