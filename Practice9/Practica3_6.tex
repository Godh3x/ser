\documentclass[11pt]{article}
\usepackage[utf8]{inputenc}

%%% PAGE DIMENSIONS
\usepackage{geometry}
\geometry{a4paper}

\usepackage{graphicx}

%%% PACKAGES
\usepackage{booktabs}
\usepackage{paralist}
\usepackage{verbatim}
\usepackage{subfig}
\usepackage{chngcntr}
\usepackage{tikz}
\usepackage[colorlinks = true,
            linkcolor = black,
            urlcolor  = blue,
            citecolor = blue,
            anchorcolor = blue]{hyperref}
\usepackage[spanish]{cleveref}
\usepackage{placeins}
\usepackage{float}
\usepackage{listings}

%%% HEADERS & FOOTERS
\usepackage{fancyhdr}
\pagestyle{fancy}
\renewcommand{\headrulewidth}{0pt}
\lhead{}\chead{}\rhead{}
\lfoot{}\cfoot{\thepage}\rfoot{}

%%% SECTION TITLE APPEARANCE
\usepackage{sectsty}
\allsectionsfont{\sffamily\mdseries\upshape}

%%% ToC (table of contents) APPEARANCE
\usepackage[nottoc,notlof,notlot]{tocbibind} % Put the bibliography in the ToC
\usepackage[titles,subfigure]{tocloft} % Alter the style of the Table of Contents
\renewcommand{\cftsecfont}{\rmfamily\mdseries\upshape}
\renewcommand{\cftsecpagefont}{\rmfamily\mdseries\upshape} % No bold!


\graphicspath{ {images/} }

\counterwithin*{figure}{section}
\counterwithin*{figure}{subsection}
\counterwithin*{figure}{subsubsection}

\counterwithin*{table}{section}
\counterwithin*{table}{subsection}
\counterwithin*{table}{subsubsection}

\addtolength{\cftfignumwidth}{2em}

\renewcommand{\thefigure}{
  \ifnum\value{subsection}=0
    \thesection.\arabic{figure}
  \else
    \ifnum\value{subsubsection}=0
      \thesubsection.\arabic{figure}
    \else
      \thesubsubsection.\arabic{figure}
    \fi
  \fi
}

\renewcommand{\thetable}{
  \ifnum\value{subsection}=0
    \thesection.\arabic{table}
  \else
    \ifnum\value{subsubsection}=0
      \thesubsection.\arabic{table}
    \else
      \thesubsubsection.\arabic{table}
    \fi
  \fi
}

%%% END Article customizations

%%% The "real" document content comes below...

\title{\Large Seguridad en Redes\\Practica 3.6}
\author{David Antuña Rodríguez\\Javier Carrión García}
\date{}

\begin{document}
  \raggedright

  \maketitle
  \newpage

  \section{OpenVPN}
    \subsection{Clave estática compartida}
      \lstset{basicstyle=\ttfamily\small}
\begin{lstlisting}
Tue Apr 24 12:11:50 2018 us=815937   shared_secret_file = 'static.key'
Tue Apr 24 12:11:50 2018 us=860926 Local Options hash (VER=V4): '8addc3e6'
Tue Apr 24 12:11:50 2018 us=860938 Expected Remote Options hash (VER=V4): '04a219ce'
Tue Apr 24 12:11:50 2018 us=860950 UDPv4 link local (bound): [undef]
Tue Apr 24 12:11:50 2018 us=860958 UDPv4 link remote: [AF_INET]192.168.1.1:1194
^[[1;5CTue Apr 24 12:11:59 2018 us=856274 Peer Connection Initiated with [AF_INET]192.168.1.1:1194
Tue Apr 24 12:12:00 2018 us=923036 Initialization Sequence Completed
^CTue Apr 24 12:18:53 2018 us=251659 event_wait : Interrupted system call (code=4)
Tue Apr 24 12:18:53 2018 us=251728 TCP/UDP: Closing socket
Tue Apr 24 12:18:53 2018 us=251761 Closing TUN/TAP interface
Tue Apr 24 12:18:53 2018 us=251787 /sbin/ifconfig tun0 0.0.0.0
Tue Apr 24 12:18:53 2018 us=264076 SIGINT[hard,] received, process exiting
\end{lstlisting}

      \begin{figure}[H]
        \centering
        \includegraphics[width = \textwidth]{tun0}
        \caption{Características de tun0.}
      \end{figure}

      \par
      Los paquetes que vemos por eth1 no se pueden leer (figura
      \ref{figure:pingeth1}), en cambio por tun0 si podemos ver su contenido
      (figura \ref{figure:pingtun0}).

      \begin{figure}[H]
        \centering
        \includegraphics[width = \textwidth]{pingeth1}
        \caption{Paquetes por eth1.}
        \label{figure:pingeth1}
      \end{figure}

      \begin{figure}[H]
        \centering
        \includegraphics[width = \textwidth]{pingtun0}
        \caption{Paquetes por tun0.}
        \label{figure:pingtun0}
      \end{figure}

    \subsection{TLS con certificados}
    \par
    Salida del comando:\\
    sudo openvpn --remote 192.168.1.1 --dev tun --ifconfig 10.4.0.2 10.4.0.1 --tls-server --dh dh1024.pem --ca ca.crt --cert server.crt --key server.key --verb 4

\begin{lstlisting}
Sat Apr 21 20:07:46 2018 us=825293 Diffie-Hellman initialized with
1024 bit key
Sat Apr 21 20:07:46 2018 us=825524 WARNING: file 'server.key' is
group or others accessible
Sat Apr 21 20:07:46 2018 us=825784 Control Channel MTU parms
[ L:1541 D:138 EF:38 EB:0 ET:0 EL:0 ]
Sat Apr 21 20:07:46 2018 us=825848 Socket Buffers:
R=[229376->131072] S=[229376->131072]
Sat Apr 21 20:07:46 2018 us=826308 TUN/TAP device tun0 opened
Sat Apr 21 20:07:46 2018 us=826320 TUN/TAP TX queue length set
to 100
Sat Apr 21 20:07:46 2018 us=826328 do_ifconfig, tt->ipv6=0,
tt->did_ifconfig_ipv6_setup=0
Sat Apr 21 20:07:46 2018 us=826340 /sbin/ifconfig tun0
10.4.0.2 pointopoint 10.4.0.1 mtu 1500
Sat Apr 21 20:07:46 2018 us=827384 Data Channel MTU parms
[ L:1541 D:1450 EF:41 EB:4 ET:0 EL:0 ]
Sat Apr 21 20:07:46 2018 us=827400 Local Options String:
'V4,dev-type tun,link-mtu 1541,tun-mtu 1500,proto UDPv4,
ifconfig 10.4.0.1 10.4.0.2,cipher BF-CBC,auth SHA1,
keysize128,key-method 2,tls-server'
Sat Apr 21 20:07:46 2018 us=827404 Expected Remote Options
String: 'V4,dev-type tun,link-mtu 1541,tun-mtu 1500,
proto UDPv4,ifconfig 10.4.0.2 10.4.0.1,cipher BF-CBC,
auth SHA1,keysize 128,key-method 2,tls-client'
Sat Apr 21 20:07:46 2018 us=827415 Local Options hash
(VER=V4): 'bd0285da'
Sat Apr 21 20:07:46 2018 us=827420 Expected Remote Options
hash (VER=V4): '599bc3b6'
Sat Apr 21 20:07:46 2018 us=827425 UDPv4 link local (bound):
[undef]
Sat Apr 21 20:07:46 2018 us=827429 UDPv4 link remote:
[AF_INET]192.168.1.1:1194
Sat Apr 21 20:07:46 2018 us=827755 TLS: Initial packet from
[AF_INET]192.168.1.1:1194, sid=77b8cb72 79f522af
Sat Apr 21 20:07:46 2018 us=835973 VERIFY OK: depth=1,
/C=KG/ST=NA/L=BISHKEK/O=OpenVPN-TEST/emailAddress=me@myhost.mydomain
Sat Apr 21 20:07:46 2018 us=836137 VERIFY OK: depth=0,
/C=KG/ST=NA/O=OpenVPN-TEST/CN=Test-Client/emailAddress=me@myhost.mydomain
Sat Apr 21 20:07:46 2018 us=845402 Data Channel Encrypt:
Cipher 'BF-CBC' initialized with 128 bit key
Sat Apr 21 20:07:46 2018 us=845452 Data Channel Encrypt:
Using 160 bit message hash 'SHA1' for HMAC authentication
Sat Apr 21 20:07:46 2018 us=845501 Data Channel Decrypt:
Cipher 'BF-CBC' initialized with 128 bit key
Sat Apr 21 20:07:46 2018 us=845524 Data Channel Decrypt:
Using 160 bit message hash 'SHA1' for HMAC authentication
Sat Apr 21 20:07:46 2018 us=846128 Control Channel: TLSv1,
cipher TLSv1/SSLv3 DHE-RSA-AES256-SHA, 2048 bit RSA
Sat Apr 21 20:07:46 2018 us=846170 [Test-Client] Peer
Connection Initiated with [AF_INET]192.168.1.1:1194
Sat Apr 21 20:07:48 2018 us=70028 Initialization
Sequence Completed
\end{lstlisting}

      \bigskip
      \par
      Para configurar la VPN cliente-servidor hemos modificado el archivo left,
      configurandolo como cliente.
      \begin{lstlisting}
      client

      dev tun
      proto tcp
      remote 192.168.1.2 1194

      ca ca.crt
      cert client.crt
      key client.key

      remote-cert-tls server
      tls-remote Test-Server
      \end{lstlisting}

      \bigskip
      \par
      Y right lo hemos configurado como servidor.
      \begin{lstlisting}
      local 192.168.1.2
      port 1194
      proto tcp

      dev tun

      ca ca.crt
      cert server.crt
      key server.key

      dh dh2048.pem

      server 10.8.0.0 255.255.255.0

      ifconfig-pool-persist ipp.txt
      \end{lstlisting}


      \bigskip
      \par
      Una vez iniciada la VPN y aplicado el filtro en Wireshark vemos los
      siguientes mensajes, figura \ref{figure:tls}. En primer lugar el cliente
      saluda al servidor para inciar la conexión y este le contesta enviando
      sus datos de autenticación. Una vez autenticado el cliente envia sus datos
      y el servidor contesta enviando la información de la sesión.

      \begin{figure}[H]
        \centering
        \includegraphics[width = \textwidth]{tls}
        \caption{Acuerdo TLS.}
        \label{figure:tls}
      \end{figure}


      \par
      Se puede escoger entre 45 conjuntos distintos, figura
      \ref{figure:ciphers}, de los cuales finalmente escogen solo uno que se
      puede ver en la figura \ref{figure:cipher}.

      \begin{figure}[H]
        \centering
        \includegraphics[width = \textwidth]{ciphers}
        \caption{Conjuntos de algoritmos.}
        \label{figure:ciphers}
      \end{figure}

      \begin{figure}[H]
        \centering
        \includegraphics[width = \textwidth]{cipher}
        \caption{Conjunto escogido.}
        \label{figure:cipher}
      \end{figure}

      \par
      En certificate el cliente envia un certificado firmado que contiene su
      clave pública.


  \section{OpenSSH}
    \subsection{Autentificación con clave pública}
      \par
      La salida del comando ssh -v 192.168.1.2 es la siguiente.

\begin{lstlisting}
OpenSSH_6.0p1 Debian-4+deb7u7, OpenSSL 1.0.1e 11 Feb 2013
debug1: Reading configuration data /etc/ssh/ssh_config
debug1: /etc/ssh/ssh_config line 19: Applying options for *
debug1: Connecting to 192.168.1.2 [192.168.1.2] port 22.
debug1: Connection established.
debug1: identity file /home/usuario/.ssh/id_rsa type 1
debug1: Checking blacklist file /usr/share/ssh/blacklist.RSA-2048
debug1: Checking blacklist file /etc/ssh/blacklist.RSA-2048
debug1: identity file /home/usuario/.ssh/id_rsa-cert type -1
debug1: identity file /home/usuario/.ssh/id_dsa type -1
debug1: identity file /home/usuario/.ssh/id_dsa-cert type -1
debug1: identity file /home/usuario/.ssh/id_ecdsa type -1
debug1: identity file /home/usuario/.ssh/id_ecdsa-cert type -1
debug1: Remote protocol version 2.0, remote software version
  OpenSSH_6.0p1 Debian-4+deb7u7
debug1: match: OpenSSH_6.0p1 Debian-4+deb7u7 pat OpenSSH*
debug1: Enabling compatibility mode for protocol 2.0
debug1: Local version string SSH-2.0-OpenSSH_6.0p1 Debian-4+deb7u7
debug1: SSH2_MSG_KEXINIT sent
debug1: SSH2_MSG_KEXINIT received
debug1: kex: server->client aes128-ctr hmac-md5 none
debug1: kex: client->server aes128-ctr hmac-md5 none
debug1: sending SSH2_MSG_KEX_ECDH_INIT
debug1: expecting SSH2_MSG_KEX_ECDH_REPLY
debug1: Server host key: ECDSA c5:9d:97:b8:6e:87:e4:e3:cc:ec:3b:a8:bc:9e:8b:12
debug1: Host '192.168.1.2' is known and matches the ECDSA host key.
debug1: Found key in /home/usuario/.ssh/known_hosts:1
debug1: ssh_ecdsa_verify: signature correct
debug1: SSH2_MSG_NEWKEYS sent
debug1: expecting SSH2_MSG_NEWKEYS
debug1: SSH2_MSG_NEWKEYS received
debug1: SSH2_MSG_SERVICE_REQUEST sent
debug1: SSH2_MSG_SERVICE_ACCEPT received
debug1: Authentications that can continue: publickey,password
debug1: Next authentication method: publickey
debug1: Offering RSA public key: /home/usuario/.ssh/id_rsa
debug1: Server accepts key: pkalg ssh-rsa blen 279
debug1: key_parse_private_pem: PEM_read_PrivateKey failed
debug1: read PEM private key done: type <unknown>
Enter passphrase for key '/home/usuario/.ssh/id_rsa':
debug1: read PEM private key done: type RSA
debug1: Authentication succeeded (publickey).
Authenticated to 192.168.1.2 ([192.168.1.2]:22).
debug1: channel 0: new [client-session]
debug1: Requesting no-more-sessions@openssh.com
debug1: Entering interactive session.
debug1: Sending environment.
debug1: Sending env LANG = es_ES.UTF-8
Linux debian 3.2.0-4-amd64 #1 SMP Debian 3.2.63-2 x86_64

The programs included with the Debian GNU/Linux system are free software;
the exact distribution terms for each program are described in the
individual files in /usr/share/doc/*/copyright.

Debian GNU/Linux comes with ABSOLUTELY NO WARRANTY, to the extent
permitted by applicable law.
Last login: Sat Apr 21 19:29:26 2018
\end{lstlisting}

    \subsection{Reenvío de puertos}
      \par
      \textbf{ssh -v -N -L 8080:www.ucm.es:80 usuario@192.168.1.2}, se ha conetado a la pagina principal
      de la complutense.

\begin{lstlisting}
Enter passphrase for key '/home/usuario/.ssh/id_rsa': 
debug1: read PEM private key done: type RSA
debug1: Authentication succeeded (publickey).
Authenticated to 192.168.1.2 ([192.168.1.2]:22).
debug1: Local connections to LOCALHOST:8080 forwarded to remote address www.ucm.es:80
debug1: Local forwarding listening on ::1 port 8080.
debug1: channel 0: new [port listener]
debug1: Local forwarding listening on 127.0.0.1 port 8080.
debug1: channel 1: new [port listener]
debug1: Requesting no-more-sessions@openssh.com
debug1: Entering interactive session.
debug1: Connection to port 8080 forwarding to www.ucm.es port 80 requested.
debug1: channel 2: new [direct-tcpip]
debug1: Connection to port 8080 forwarding to www.ucm.es port 80 requested.
debug1: channel 3: new [direct-tcpip]
debug1: Connection to port 8080 forwarding to www.ucm.es port 80 requested.
debug1: channel 4: new [direct-tcpip]
debug1: Connection to port 8080 forwarding to www.ucm.es port 80 requested.
debug1: channel 5: new [direct-tcpip]
debug1: Connection to port 8080 forwarding to www.ucm.es port 80 requested.
debug1: channel 6: new [direct-tcpip]
debug1: Connection to port 8080 forwarding to www.ucm.es port 80 requested.
debug1: channel 7: new [direct-tcpip]
\end{lstlisting}

      \bigskip
      \par
      \textbf{ssh -v -X -R 8080:www.ucm.es:80 usuario@192.168.1.2 chromium}\\
      Se abre en left el programa, en este caso chromium, por ser la que ejecuta el comando. El
      puerto 8080 que está escuchando es el de right.

\begin{lstlisting}
Enter passphrase for key '/home/usuario/.ssh/id_rsa': 
debug1: read PEM private key done: type RSA
debug1: Authentication succeeded (publickey).
Authenticated to 192.168.1.2 ([192.168.1.2]:22).
debug1: Remote connections from LOCALHOST:8080 forwarded to local address www.ucm.es:80
debug1: channel 0: new [client-session]
debug1: Requesting no-more-sessions@openssh.com
debug1: Entering interactive session.
debug1: remote forward success for: listen 8080, connect www.ucm.es:80
debug1: All remote forwarding requests processed
debug1: Requesting X11 forwarding with authentication spoofing.
debug1: Sending environment.
debug1: Sending env LANG = es_ES.UTF-8
debug1: Sending command: chromium
debug1: client_input_channel_open: ctype x11 rchan 5 win 65536 max 16384
debug1: client_request_x11: request from ::1 41802
debug1: channel 1: new [x11]
debug1: confirm x11
debug1: client_input_channel_open: ctype x11 rchan 6 win 65536 max 16384
debug1: client_request_x11: request from ::1 41803
debug1: channel 2: new [x11]
debug1: confirm x11
debug1: client_input_channel_open: ctype x11 rchan 7 win 65536 max 16384
debug1: client_request_x11: request from ::1 41804
debug1: channel 3: new [x11]
debug1: confirm x11
OpenGL Warning: Failed to connect to host. Make sure 3D acceleration is enabled for this VM.
debug1: client_input_channel_open: ctype forwarded-tcpip rchan 8 win 2097152 max 32768
debug1: client_request_forwarded_tcpip: listen localhost port 8080, originator ::1 port 37157
debug1: connect_next: host www.ucm.es ([147.96.1.15]:80) in progress, fd=10
debug1: channel 4: new [::1]
debug1: confirm forwarded-tcpip
debug1: client_input_channel_open: ctype forwarded-tcpip rchan 9 win 2097152 max 32768
debug1: client_request_forwarded_tcpip: listen localhost port 8080, originator ::1 port 37158
debug1: connect_next: host www.ucm.es ([147.96.1.15]:80) in progress, fd=11
debug1: channel 5: new [::1]
debug1: confirm forwarded-tcpip
debug1: channel 4: connected to www.ucm.es port 80
debug1: channel 5: connected to www.ucm.es port 80
debug1: client_input_channel_open: ctype forwarded-tcpip rchan 10 win 2097152 max 32768
debug1: client_request_forwarded_tcpip: listen localhost port 8080, originator ::1 port 37159
debug1: connect_next: host www.ucm.es ([147.96.1.15]:80) in progress, fd=12
debug1: channel 6: new [::1]
debug1: confirm forwarded-tcpip
debug1: client_input_channel_open: ctype forwarded-tcpip rchan 11 win 2097152 max 32768
debug1: client_request_forwarded_tcpip: listen localhost port 8080, originator ::1 port 37160
debug1: connect_next: host www.ucm.es ([147.96.1.15]:80) in progress, fd=13
debug1: channel 7: new [::1]
debug1: confirm forwarded-tcpip
debug1: client_input_channel_open: ctype forwarded-tcpip rchan 12 win 2097152 max 32768
debug1: client_request_forwarded_tcpip: listen localhost port 8080, originator ::1 port 37161
debug1: connect_next: host www.ucm.es ([147.96.1.15]:80) in progress, fd=14
debug1: channel 8: new [::1]
debug1: confirm forwarded-tcpip
debug1: client_input_channel_open: ctype forwarded-tcpip rchan 13 win 2097152 max 32768
debug1: client_request_forwarded_tcpip: listen localhost port 8080, originator ::1 port 37162
debug1: connect_next: host www.ucm.es ([147.96.1.15]:80) in progress, fd=15
debug1: channel 9: new [::1]
debug1: confirm forwarded-tcpip
debug1: channel 6: connected to www.ucm.es port 80
debug1: channel 7: connected to www.ucm.es port 80
debug1: channel 8: connected to www.ucm.es port 80
debug1: channel 9: connected to www.ucm.es port 80
\end{lstlisting}

      \bigskip
      \par
      \textbf{ssh -v -N -D 1080 usuario@192.168.1.2}\\
      Para el servidor la maquina que quiere conectarse es right, que es la que
      hace de proxy.

\begin{lstlisting}
Enter passphrase for key '/home/usuario/.ssh/id_rsa': 
debug1: read PEM private key done: type RSA
debug1: Authentication succeeded (publickey).
Authenticated to 192.168.1.2 ([192.168.1.2]:22).
debug1: Local connections to LOCALHOST:1080 forwarded to remote address socks:0
debug1: Local forwarding listening on ::1 port 1080.
debug1: channel 0: new [port listener]
debug1: Local forwarding listening on 127.0.0.1 port 1080.
debug1: channel 1: new [port listener]
debug1: Requesting no-more-sessions@openssh.com
debug1: Entering interactive session.
debug1: Connection to port 1080 forwarding to socks port 0 requested.
debug1: channel 2: new [dynamic-tcpip]
debug1: Connection to port 1080 forwarding to socks port 0 requested.
debug1: channel 3: new [dynamic-tcpip]
debug1: Connection to port 1080 forwarding to socks port 0 requested.
debug1: channel 4: new [dynamic-tcpip]
debug1: Connection to port 1080 forwarding to socks port 0 requested.
debug1: channel 5: new [dynamic-tcpip]
debug1: Connection to port 1080 forwarding to socks port 0 requested.
debug1: channel 6: new [dynamic-tcpip]
debug1: Connection to port 1080 forwarding to socks port 0 requested.
debug1: channel 7: new [dynamic-tcpip]
debug1: Connection to port 1080 forwarding to socks port 0 requested.
debug1: channel 8: new [dynamic-tcpip]
debug1: Connection to port 1080 forwarding to socks port 0 requested.
debug1: channel 9: new [dynamic-tcpip]
debug1: Connection to port 1080 forwarding to socks port 0 requested.
debug1: channel 10: new [dynamic-tcpip]
debug1: Connection to port 1080 forwarding to socks port 0 requested.
debug1: channel 11: new [dynamic-tcpip]
debug1: Connection to port 1080 forwarding to socks port 0 requested.
debug1: channel 12: new [dynamic-tcpip]
debug1: Connection to port 1080 forwarding to socks port 0 requested.
debug1: channel 13: new [dynamic-tcpip]
debug1: Connection to port 1080 forwarding to socks port 0 requested.
debug1: channel 14: new [dynamic-tcpip]
debug1: Connection to port 1080 forwarding to socks port 0 requested.
debug1: channel 15: new [dynamic-tcpip]
debug1: Connection to port 1080 forwarding to socks port 0 requested.
debug1: channel 16: new [dynamic-tcpip]
debug1: Connection to port 1080 forwarding to socks port 0 requested.
debug1: channel 17: new [dynamic-tcpip]
channel 15: open failed: administratively prohibited: open failed
debug1: channel 15: free: direct-tcpip: listening port 1080 for xcsvmqbwrjy port 80, connect from ::1 port 51147, nchannels 18
channel 16: open failed: administratively prohibited: open failed
channel 17: open failed: administratively prohibited: open failed
debug1: channel 16: free: direct-tcpip: listening port 1080 for kpkchnuhbfv port 80, connect from ::1 port 51148, nchannels 17
debug1: channel 17: free: direct-tcpip: listening port 1080 for urhtjlxedlzo port 80, connect from ::1 port 51149, nchannels 16
debug1: Connection to port 1080 forwarding to socks port 0 requested.
debug1: channel 15: new [dynamic-tcpip]
debug1: Connection to port 1080 forwarding to socks port 0 requested.
debug1: channel 16: new [dynamic-tcpip]
debug1: Connection to port 1080 forwarding to socks port 0 requested.
debug1: channel 17: new [dynamic-tcpip]
debug1: Connection to port 1080 forwarding to socks port 0 requested.
debug1: channel 18: new [dynamic-tcpip]
debug1: Connection to port 1080 forwarding to socks port 0 requested.
debug1: channel 19: new [dynamic-tcpip]
debug1: Connection to port 1080 forwarding to socks port 0 requested.
debug1: channel 20: new [dynamic-tcpip]
debug1: Connection to port 1080 forwarding to socks port 0 requested.
debug1: channel 21: new [dynamic-tcpip]
debug1: Connection to port 1080 forwarding to socks port 0 requested.
debug1: channel 22: new [dynamic-tcpip]
debug1: Connection to port 1080 forwarding to socks port 0 requested.
debug1: channel 23: new [dynamic-tcpip]
debug1: Connection to port 1080 forwarding to socks port 0 requested.
debug1: channel 24: new [dynamic-tcpip]
debug1: channel 8: free: direct-tcpip: listening port 1080 for www.google.es port 443, connect from ::1 port 51140, nchannels 25
debug1: channel 9: free: direct-tcpip: listening port 1080 for www.google.es port 443, connect from ::1 port 51141, nchannels 24
debug1: channel 10: free: direct-tcpip: listening port 1080 for www.google.es port 443, connect from ::1 port 51142, nchannels 23
debug1: channel 11: free: direct-tcpip: listening port 1080 for www.google.es port 443, connect from ::1 port 51143, nchannels 22
debug1: channel 4: free: direct-tcpip: listening port 1080 for apis.google.com port 443, connect from ::1 port 51136, nchannels 21
debug1: channel 5: free: direct-tcpip: listening port 1080 for ssl.gstatic.com port 443, connect from ::1 port 51137, nchannels 20
debug1: channel 6: free: direct-tcpip: listening port 1080 for ssl.gstatic.com port 443, connect from ::1 port 51138, nchannels 19
debug1: channel 12: free: direct-tcpip: listening port 1080 for www.gstatic.com port 443, connect from ::1 port 51144, nchannels 18
\end{lstlisting}

\end{document}
