\documentclass[11pt]{article}
\usepackage[utf8]{inputenc}

%%% PAGE DIMENSIONS
\usepackage{geometry}
\geometry{a4paper}

\usepackage{graphicx}

%%% PACKAGES
\usepackage{booktabs}
\usepackage{paralist}
\usepackage{verbatim}
\usepackage{subfig}
\usepackage{chngcntr}
\usepackage{tikz}
\usepackage[colorlinks = true,
            linkcolor = black,
            urlcolor  = blue,
            citecolor = blue,
            anchorcolor = blue]{hyperref}
\usepackage[spanish]{cleveref}
\usepackage{placeins}
\usepackage{float}

%%% HEADERS & FOOTERS
\usepackage{fancyhdr}
\pagestyle{fancy}
\renewcommand{\headrulewidth}{0pt}
\lhead{}\chead{}\rhead{}
\lfoot{}\cfoot{\thepage}\rfoot{}

%%% SECTION TITLE APPEARANCE
\usepackage{sectsty}
\allsectionsfont{\sffamily\mdseries\upshape}

%%% ToC (table of contents) APPEARANCE
\usepackage[nottoc,notlof,notlot]{tocbibind} % Put the bibliography in the ToC
\usepackage[titles,subfigure]{tocloft} % Alter the style of the Table of Contents
\renewcommand{\cftsecfont}{\rmfamily\mdseries\upshape}
\renewcommand{\cftsecpagefont}{\rmfamily\mdseries\upshape} % No bold!


\graphicspath{ {images/} }

\counterwithin*{figure}{section}
\counterwithin*{figure}{subsection}
\counterwithin*{figure}{subsubsection}

\counterwithin*{table}{section}
\counterwithin*{table}{subsection}
\counterwithin*{table}{subsubsection}

\addtolength{\cftfignumwidth}{2em}

\renewcommand{\thefigure}{
  \ifnum\value{subsection}=0
    \thesection.\arabic{figure}
  \else
    \ifnum\value{subsubsection}=0
      \thesubsection.\arabic{figure}
    \else
      \thesubsubsection.\arabic{figure}
    \fi
  \fi
}

\renewcommand{\thetable}{
  \ifnum\value{subsection}=0
    \thesection.\arabic{table}
  \else
    \ifnum\value{subsubsection}=0
      \thesubsection.\arabic{table}
    \else
      \thesubsubsection.\arabic{table}
    \fi
  \fi
}

%%% END Article customizations

%%% The "real" document content comes below...

\title{\Large Seguridad en Redes\\Practica 2.3}
\author{David Antuña Rodríguez\\Javier Carrión García}
\date{}

\begin{document}
  \raggedright

  \maketitle
  \newpage

  \section{OpenSSL}
    \subsection{Creación de una CA}
      \par
      Si intentamos utilizar otros directorios dará error a no ser que se modifique el fichero de
      configuración, concretamente los datos de la sección [CA\_default].

      \bigskip
      \par
      La clave privada se ha almacenado en \textit{demoCA/private/cakey.pem} y el certificado en
      \textit{demoCA/cacert.pem}. Como hemos empleado la opción -new la clave generada es de tipo
      RSA y su tamaño viene especificado en el fichero de configuracion, valor de default\_bits
      en la sección [req], por defecto son 2048 bits.

      \bigskip
      \par
      Los ficheros de este apartado están en la carpeta \textit{ca},

    \subsection{Creación de solicitudes de firma de certificado}
      \par
      El algoritmo por defecto se encuentra en la variable default\_md del fichero de configuración,
      en nuestro caso hemos empleado la opción -new que crea una clave rsa nueva y la ha almacenado
      en userkey1.pem y userkey2.pem, también podriamos haber utilizado una que tuvieramos
      previamente con la opción -key.

      \bigskip
      \par
      El certificado contiene los datos que identifican al usuario solicitante, incluida
      su clave pública. Puede verse en los ficheros de la carpeta \textit{csr}

    \subsection{Consulta y manipulación de certificados}
      \par
      La información de los certificados se puede ver en los ficheros de la carpeta \textit{certs}. Los
      certificados usan la version 3.

      \bigskip
      \par
      En un certificado van todos los datos del poseedor del certificado, Subject, junto con la
      validez del mismo.

      \bigskip
      \par
      Si se quiere comprobar si el poseedor es una CA basta con ver el common name del Subject,
      x509 también añade un campo Basic Constraints en los certificados que no son de CA con el
      valor CA:FALSE.

    \subsection{Revocación de certificados}
      \par
      La crl está en la carpeta \textit{revocacion}.


      \bigskip
      \par
      Hay un certificado revocado con serial number 02, corresponde a usercert2, lo ha revocado la CA que
      es quien ha creado el crl, campo Issuer. El CRL es válido porque la CA lo ha firmado.

  \section{GnuPG}
    \subsection{Firma de claves (Web of trust)}
      \par
      En los check-trust y validity aparecen claves 2 extra, con nivel ultimate, que corresponden a las dos que generamos en
      la práctica anterior. Tambien está en el anillo la clave pública que subiste el otro día, tiene confianza desconocida.

      \bigskip
      \par
      Todos los ficheros están en la carpeta \textit{trust}.

      \bigskip
      \par
      C será valida porque aunque desconocemos la validez de C sí conocemos la de B, y nos
      fiamos de ella, y la clave C ha sido firmada por B.\\
      Ficheros \textit{checktrustdb.txt} y \textit{validity.txt}.

      \bigskip
      \par
      En el segundo caso C no será válida porque B es dudosa y para validar una clave con firmas
      dudosas son necesarias al menos 3, también debe cumplir que el camino que lleva hasta nuestra
      firma es menor de 5, como hemos firmado B con la clave A esto se cumple pero estas condiciones
      no son excluyentes.\\
      Ficheros \textit{checktrustdb2.txt} y \textit{validity2.txt}.

      \bigskip
      \par
      Ahora C vuelve a ser válida porque como hemos explicado antes ha sido firmada por 3 claves
      dudosas (B, D y E) y además el camino a nuestra clave es menor de 5.\\
      Ficheros \textit{checktrustdb3.txt} y \textit{validity3.txt}.
    \end{document}
